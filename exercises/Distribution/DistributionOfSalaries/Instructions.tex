\documentclass[a4,11pt]{article}%
\usepackage{amsmath,amsfonts,amsthm}% ams packages
\usepackage[svgnames]{xcolor}% named colors
\usepackage[hang,small,labelfont=bf,up,textfont=it,up]{caption}% caption
\usepackage{booktabs}% tables

\title{Application: The Distribution of Incomes in Competitive Sport}
\author{Patrick Toche}
\date{}

\begin{document}
 \pagenumbering{gobble}
\maketitle

\newpage
 \pagenumbering{gobble}


\subsection*{The Distribution of Incomes in Competitive Sport}

The purpose of this project is to explore the distribution of income among top professional athletes. Select a professional team in a collective sport, e.g. male soccer, female basketball. Alternatively, select an individual sport like female tennis or male golf. Gather data about the salaries and bonuses earned by at least 10 members of the same team or by the top 10 ranked athletes, for a single year, e.g 2016. Do not mix sport. Do not mix gender. Do not include capital income. Do include endorsements if you can.

To illustrate the type of data you are expected to put together, a data file of the Real Madrid football team is available [xls], as obtained at the start of the 2014-2015 season. Do not choose Real Madrid for your project.

\setlength{\leftmargini}{1em}
\begin{enumerate}

\item 
Collect the data in a worksheet with the names listed vertically and corresponding income listed in the next column. Create a plot --- or several plots --- that you feel best illustrate the key features of the distribution of income. Make clear graphs with informative labels. Summarize the key points in a caption below the figure. Print each figure by itself on an A4 page. If your data sources and computations take more than a few lines, print them on a separate page. 

\item 
Select the best paid athlete in your list and estimate his/her total income over their lifetime, assuming they will live for another $50$ years, work for another $10$ years, and retire into anonymity afterwards. Assume they earn a rate of return $r$, pay a percentage $t$ in taxes, and save a percentage $s$. Pick numbers you feel are realistic and consider alternative scenarios. How much wealth would they have accumulated over that period? Explain your calculations.

\item 
Plot the ratio of consumption to wealth over their remaining lifetime. Plot the share of capital income over time. Explore alternative assumptions about $r$, $s$, $t$, and discuss the sensitivity of your answers to other values.

\item 
Is a constant saving rate $s$ over a lifetime likely to be optimal? Consider other assumptions and discuss their impact on the accumulation of wealth. 

\item
In a separate text document, clearly indicate the data source (website, news article); the currency used; any calculations you have made (currency conversion, translation from monthly to annual, etc.); whether the data includes endorsement revenues; any other relevant information. 

\end{enumerate}


\end{document}